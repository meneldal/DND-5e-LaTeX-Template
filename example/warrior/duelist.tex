\begin{path}
{Duelist}
{AGI (Actions) / STR (Melee Attacks)}
{Duelists wield their weapons with skill and elegance. Whether its rapiers, pistols, or their screaming enemies, duelists understand there's more to weapons than just swinging them around.}
{Duelists excel at melee combat, though skills like Dual Wield and Quick Draw are also useful for ranged weapons. The Mind Slice skill allows them to target an enemy's weakness and damage any stat. Finally, their ability to Cut Fighting Spirit allows them to end fights before they even really begin.}
\skilldescription
{Cut Fighting Spirit}
{8}
{A technique developed by the goddess of swords, Blade. Focus power into your weapon and strike without leaving a scratch, causing the enemy to lose their will to fight you.
\\Add the stat of your choice to a normal weapon attack vs. the target's WIL. On a success, the target takes 1 WIL wound and loses the desire to fight. Effect ends if they are attacked, as they will still fight in self-defense.}
{A non-lethal strike that causes the enemy to lose the will to fight.}
\skilldescription
{Dual Wield}
{7}
{When wielding a weapon in each hand, you can attack with both weapons. You can also use this skill when wielding a weapon and shield, or while unarmed to attack with both fists.
\\Spend a single action to attack with both weapons at a -10 penalty. You must roll each attack separately and the penalty applies to both rolls. You can attack more than twice, but the penalty increases by -5 for each subsequent attack.
\\\textbf{EXAMPLE:} If you're an asura with six arms, wielding a weapon in each hand, you can attack with all six arms on a single action, taking a -30 penalty to each roll.}
{Take increasing penalty to attack multiple times when dual wielding.}
\skilldescription
{Duel}
{6}
{Challenge a single target to a duel to gain bonuses against your opponent. Costs 1 action to challenge someone. If they accept, both participants must focus on each other and cannot attack other targets.
\\Participants gain +1 wound and +10 to all attacks during the duel. However, participants take a -10 penalty on defense rolls against attacks from anyone other than their opponent. Participants get a free attack if their opponent flees or attacks an outside target.
\\The duel ends when a specified condition is met, such as first blood or death. The duel automatically ends if either party becomes unable to continue fighting due to unconsciousness, death, immobility, or some other circumstance.}
{Challenge enemies to a duel, gaining bonuses during the encounter.}
\skilldescription
{Feint}
{6}
{By changing the direction of your swing at the last moment, you're able to create an opening in your opponent's defenses.
\\Roll AGI vs. SEN. On a success, your next attack against the same target cannot be opposed. The opponent cannot roll defense and your attack simply must beat their passive armor bonus.}
{Use misdirection to create an opening to attack.}
\skilldescription
{Foe Hammer}
{5}
{Use the struggling bodies of your enemies as improvised weapons. Add your STR (SB) to opposed STR rolls to maintain your grip on the enemy. Enemies used as weapons are considered different ranks relative to their size. An average adult human is an A-rank weapon. Enemies who are being wielded take wounds equal to the number of wounds they are used to inflict.}
{Grab your enemies and use their bodies as improvised weapons.}
\skilldescription
{Mind Slice}
{8}
{With focus, your blade can cut through almost anything.
\\Costs 1 MAG to activate. Spend 1 action to select a target. Your attacks against that target deal an additional +1 wound to the stat of your choice. This effect lasts until the target dies or you select a new target.}
{Attacks deal +1 wound to stat of choice against selected target.}
\skilldescription
{Monkey Grip}
{3}
{If something is worth doing, it's worth overdoing. You can wield two-handed weapons one-handed with no penalty. This does not affect weapons that require two hands to operate, such as a bow and arrow. You can dual wield with a B-rank or A-rank weapon with no penalty. You can also wield oversized weapons at no penalty. For example, you could wield a greatsword with one hand, or a child-sized creature could wield a weapon made for an adult-sized creature.
\\\textbf{NOTE:} Normally, you would take a penalty to attack rolls by wielding two-handed weapons with one hand or attempting to use oversized weapons. You also cannot dual wield B-rank or A-rank weapons without incurring a penalty. The exact penalty is subject to GM discretion, but is typically -5 or -10.}
{Wield oversized and two-handed weapons with 1 hand.}
\skilldescription
{Parry}
{3}
{Even without a shield, you can block blows with careful movements and well placed strikes. You can use your weapon as a shield 1 size smaller to block blows.
\\\textbf{EXAMPLE:} an A-rank weapon can be used as a B-rank shield.}
{Use a weapon as a shield 1 size smaller.}
\skilldescription
{Quick Draw}
{3}
{By drawing your weapon in an instant, you can get the jump on your enemies. You can draw weapons in combat without spending an action to do so. You also gain a +5 bonus on opposed AGI rolls for quick draw contests.
\\\textbf{NOTE:} Drawing a sheathed or stowed weapon normally requires 1 action.}
{Draw weapon without spending an action.}
\skilldescription
{Riposte}
{7}
{With careful timing, you can exploit a momentary opening after your enemy strikes. After a successful parry, you can make an immediate free attack with a +10 bonus to hit.}
{After successful parry, make an immediate free attack.}
\skilldescription
{Spin Attack}
{5}
{Spin in a circle, striking all adjacent enemies. Take a -10 penalty to the attack to hit all enemies in range with a single attack roll.}
{Take a -10 penalty on attack to target all adjacent creatures.}

\end{path}
