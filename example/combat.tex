\chapter{Combat}
\begin{multicols}{2}
\section{Basics of Combat}
If there's one thing adventurers love to do, it's kill things for fun and profit. Here's how you do it.
\subsection{Attack and Defense}
\begin{itemize}
\item The attacker rolls the relevant rank dice and adds the relevant bonuses.
\item The defender can choose to block or dodge. To block, you roll your shield's rank. To dodge, you roll your AGI (RD). Add your static armor bonus to all defense rolls.
\item If the attacker succeeds, the defender takes 1 wound. They may take more if the attacker uses a skill that states the defender takes more wounds.
\end{itemize}
\subsection{Wounds}
Wounds are taken directly to your stats. Wounds are recorded under the temp stat area. Ability costs are also treated as wounds.

All attacks damage Vitality by default. The defender can choose to take wounds to their Strength or Agility if an attack would otherwise kill them.

\note{Some skills specifically target other stats.}

\subsection{Actions}
At the start of combat, all combatants are dealt a number of cards equal to their AGI rank. Each card represents an action point (AP). AP are the number of actions that a character can take in a round. Actions can be used for a variety of things such as making an attack, using a class skill, moving more than 1 meter, and much more.

\subsection{Turn Order}
When a player or GM wants to take an action, they declare what action they wish to take and lay a card down. All relevant rolls are made and then combat moves on. This continues until all parties involved have used up their actions or decided to end their turn. When everyone is out of actions, the round resets and everyone gets their actions back.

\subsection{Called Shots}
Normally, you take a -5 penalty on called shots. This increases based on the difficulty subject to GM discretion.

\section{Weapons}

Weapons have ranks and rank dice just like stats. A weapon's rank is based on its size. Add your STR (SB) to melee attacks and your AGI (SB) to ranged attacks. Examples of weapons are listed below.

\begin{wldescription}
\item [E rank:] Fists, slings, etc.
\item [D rank:] Daggers, hand crossbows, etc.
\item [C rank:] Swords, hunting bows, etc.
\item [B rank:] Maces, rifles, etc.
\item [A rank:] Greatswords, elephant guns, etc.
\end{wldescription}
It's possible for weapons to be EX rank, though this is reserved for exceptionally powerful magic weapons. GMs should exercise discretion and avoid giving out EX rank weapons except in special circumstances or very high level campaigns.

\note{Small weapons like brass knuckles or a butterfly knife may still count as E-rank weapons, but can be used to gain weapon abilities such as Crushing or Bleed. This would effectively let you use a weapon ability with your fists while still maintaining the size bonus of an E-rank weapon.}

\subsection{Weapon Abilities}
Weapons can have special abilities that apply to all attacks made with them. These abilities have ranks separate from the size rank of the weapon. Typically, lower ranked weapons have higher ranked abilities to compensate. These abilities can include the following:
\begin{wldescription}
\item [Bleed:] Target takes an additional wound for 1 round per rank. Healing negates the effect. (Example: serrated blades)
\item [Crushing:] Ignores 1 point of DR per rank. (Example: blunt weapons)
\item [Piercing:] Ignores 5 points of armor per rank. In other words, attacker gets +5 to hit per rank. (Example: thin, stabbing weapons)
\end{wldescription}
Other weapon abilities can potentially exist. One simple way of creating new weapon abilities is to add a class skill to a weapon.

Weapons may have multiple abilities, but it is recommended that the total ranks for weapons and abilities not exceed EX1 to start with. To calculate a weapon's total rank, add its size rank and the ranks of all its abilities together.

\note[EXAMPLE:]{A D-rank dagger with B-rank Bleed is an EX1 weapon. A C-rank sword with Piercing (E) and Bleed (E) is an A-rank weapon.}

\subsection{Reach}
Some weapons, such as polearms and greatswords, can hit enemies from a greater distance. Reach is expressed in meters (e.g. Reach 1m, Reach 2m, etc.).

In some cases, ranged weapons may have ranges that function as a longer version of reach. For example, a shotgun may have an effective range of 10m, but a sniper rifle may have an effective range of 1km or more. Having effective ranges for ranged weapons is optional and subject to GM discretion.

\subsection{Size Bonuses}
Smaller weapons gain a bonus to hit in certain situations. These bonuses are +10 for E-Rank weapons and +5 for D-Rank weapons. B-Rank and A-Rank weapons may take comparable penalties in the same situations subject to GM discretion. These bonuses are applied in the following situations:
\begin{itemize}
\item Throwing a melee weapon
\item Sneak attacks
\item Fighting in enclosed spaces
\item Dual-wielding
\item Any kind of rapid attack skill (e.g. “Flurry of blows”, “Rapid shot”, etc.)
\end{itemize}

\note{Normally, you cannot dual-wield with a B or A-Rank weapon without the “Monkey Grip” skill.}

\section{Armor \& Shields}

\subsection{Armor}
There are three slots for armor: Head, Torso, and Legs. These slots may be modified if you're playing a race with a non-standard anatomy. You can wear different types of armor in different slots. Armor is divided into 5 ranks with increasing bonuses based on rank.
\begin{wldescription}
\item [E rank:] +1 to defense
\item [D rank:] +2 to defense
\item [C rank:] +3 to defense
\item [B rank:] +4 to defense
\item [A rank:] +5 to defense
\end{wldescription}
Add the defense bonuses for all your armor slots together. This is your total armor bonus.
Armor may have higher bonuses or special abilities based on enchantments or other effects, subject to GM discretion.

EX rank armor grants increasing bonuses based on its rank (EX1 = +6, EX2 = +7, etc.) in addition to any special abilities it has for being EX rank armor. These bonuses may be even higher based on enchantments. EX rank armor should be used sparingly by the GM.


\subsection{Shields}
Shields function like defensive weapons. They have the same ranks and rank dice as weapons. These ranks are based on the size of the shield, from buckers (E-rank) to tower shields (A-rank). You use shields to block when defending. Shields with an EX rank function the same as weapons with an EX rank and should be used sparingly by the GM.

\note{You may also use shields to make an attack called a shield bash. Treat the shield as a weapon 2 ranks smaller. If you have the Shield Bash skill, shields are treated as weapons 1 rank smaller.}


\subsection{Relative Size Modifiers}
An enemy being significantly larger or smaller affects combat. Size difference is measured on a sliding scale. Both the attacker and defender take bonuses or penalties based on these size differences when applicable.
\needspace{4\baselineskip}
\begin{wldescription}
\item [Smaller Size:] +5 Dodge, -5 Attack per size difference.
\item [Larger Size:] +1 Wound, -5 Dodge per size difference.
\end{wldescription}

\begin{wltable}[XXXXX]<Size Values><tab:sizevalues><
Size 1  & Size 2 & Size 3 & Size 4 & Size 5>
XS & S & M & L & XL\\
Extra Small & Small & Medium & Large & Extra Large\\
\end{wltable}

Size difference is subjective and open to interpretation by the GM. Use table~\ref{tab:sizeratio} to compare size of the combatants. Degrees of separation determine their relative size.

\begin{wltable}[ll]<Degrees of Separation><tab:sizeratio><
Degrees Of Separation & Size>
1 & Half-Size\\
2 & Quarter-Size\\
3 & Fits In Mouth\\
4 & Tooth-Size\\
\end{wltable}

\section{Misc. Stuff}

\subsection{Resistances \& Weaknesses}
Players and enemies can have resistances to certain types of damage such as fire, electricity, or acid. Resistance to physical damage (such as from weapons) is called Damage Resistance, more commonly referred to as simply “DR”.
\begin{itemize}
\item You can ignore a number of wounds of that type each round equal to the amount of resistance.
\item Resistances max out at 5.
\end{itemize}
Beyond resistance is immunity. If you have immunity, you simply don’t take wounds of that type at all. You can also gain immunity to effects such as disease or poison.

Having a weakness to a particular type means you take double the usual amount of wounds. See below for examples.
\begin{wldescription}
\item [DR 3:] Ignore 3 physical wounds per round
\item [Resist Ice 1:] Ignore 1 Ice wound per round
\item [Resist Sonic 1:] Ignore 1 sonic wound per round
\item [Immune: Poison:] Ignore all poison wounds
\item [Weakness: Fire:] Take double wounds from fire
\end{wldescription}

\subsection{Buffs \& Debuffs}
Buffs are temporary increases to a stat or any temporary special effect beneficial to the character. 

Buffs may include beneficial effects, such as temporary elemental resistance, flight, or automatic translation.

Debuffs are temporary decreases to a stat or any temporary negative condition. Stat debuffs differ from wounds in that they go away on their own after a short period of time.

Debuffs may include negative status conditions, such as blindness, paralysis, or poison.

\subsection{Status Effects}
Below is a list of some potential status effects. These are just some of the potential status effects that can be inflicted.
\begin{wldescription}
\item [Blindness:] Automatically fail sight-based SEN checks. May take penalties to attack or defense rolls subject to GM discretion.
\item [Damage over Time:]  Take 1 wound at regular intervals for a set duration. Can be the result of poison, bleed, or burns.
\item [Deafness:]  Automatically fail hearing-based SEN checks. May take penalties to attack or defense rolls subject to GM discretion.
\item [Stun:] Target is unable to move for a set number of actions.
\end{wldescription}

\subsection{Stat Regeneration Rate}
Each stat recovers a number of wounds each day equal to its rank unless a condition prevents a normal regeneration from occurring.

\section{Social Combat}

In addition to physical combat, there is also social combat which represents social struggles such as arguing, manipulation, and negotiation. Just like physical combat, certain stats play a key role in social combat.

\subsection{Charisma}
When trying to convince someone to do something or believe something, you roll your CHA (RD) against the stat they choose to defend themselves with. This functions like making an attack roll in combat.

\subsection{Willpower}
WIL is the default stat used to defend against persuasion and mental effects. It also functions like your mental vitality. As you fail defensive rolls in social combat, you take wounds to you WIL, gradually losing the ability to stand up to your opponent's argument.

\subsection{Intelligence}
Optionally, you can choose to use INT to attack or defend in social combat. You can substitute your INT when you're making an argument based on fact and logic with less focus on delivering the message in a convincing way. You can use INT to defend when someone is trying to convince you of something that you know is factually untrue.

\subsection{Sense}
You can use your SEN to detect someone's intentions or whether they're being honest with you. You can also make unopposed SEN checks to see if something is a good idea.

These are just the basics of social combat. You can use certain skills, particularly from the Socialite path in Scholar, to improve your effectiveness in social combat.

\end{multicols}